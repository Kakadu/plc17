\documentclass[10pt]{article}

\usepackage[T1,T2A]{fontenc}
\usepackage[utf8]{inputenc}
\usepackage[english,russian]{babel}

\newcommand{\code}{\texttt}

\title{Независимая от компилятора библиотека точной сборки мусора для языка C++}
\author{
        Евгений Моисеенко \\
        %% Санкт-Петербургский государственный университет\\
        %% Математико-механический факультет\\
            \and
        Даниил Березун\\
        %% Санкт-Петербургский государственный университет\\
        %% Математико-механический факультет\\
}
\date{\today}

\begin{document}
\maketitle

\begin{abstract}
В данной работе представлен сборщик мусора для языка C++.
Представленный сборщик мусора является точным и не требует поддержки со стороны компилятора, 
а также поддерживает сжатие и параллельную маркировку (\textit{concurrent marking}).
Библиотека определяет класс умного указателя --- \code{gc\_ptr}.
Представленный в работе подход позволяет совмещать использование 
трассирующего сборщика мусора с другими методами управления памятью в C++, 
в том числе с ручным управлением памятью и методами основанными на использовании
умных указателей \code{std::unique\_ptr} и \code{std::shared\_ptr}. 

\end{abstract}

\section{Introduction}

Большинство современных языков программирования активно использует \emph{динамическое распределение памяти}, 
при котором выделение памяти осуществляется во время исполнения программы. 
\emph{Автоматическое управление памятью} избавляет программиста от необходимости вручную освобождать выделенную память, 
устраняя тем самым целый класс возможных ошибок и увеличивая безопасность исходного кода программы. 
\emph{Сборка мусора} давно стала стандартом в области автоматического управления памятью.


Язык C++ разрабатывался с расчетом на использование ручного управления памятью.
Добавление средств автоматического управления памятью на протяжении многих лет являлось предметом долгих дискуссий. 


\paragraph{Outline}
The remainder of this article is organized as follows.
Section~\ref{previous work} gives account of previous work.
Our new and exciting results are described in Section~\ref{results}.
Finally, Section~\ref{conclusions} gives the conclusions.

\section{Previous work}\label{previous work}
A much longer \LaTeXe{} example was written by Gil~\cite{Gil:02}.

\section{Results}\label{results}
In this section we describe the results.

\section{Conclusions}\label{conclusions}
We worked hard, and achieved very little.

\bibliographystyle{abbrv}
\bibliography{main}

\end{document}
